\section{Results and Empirical Findings}

Our experiments provide empirical evidence for the observability and governability of high-dimensional generative pipelines.

\subsection{Detection of Latent Instability}
The system demonstrates high sensitivity to latent corruption. As shown in \textbf{Figure 1}, RLCS sensors respond monotonically to gradual drift. The ResLik score tracks the deviation from the reference centroid, providing a clear signal for the \texttt{ABSTAIN} decision once the threshold $\tau_D=3.0$ is exceeded. Concurrently, the TCS sensor identifies the loss of temporal stability, which is essential for handling streaming biological data or video sequences.

\subsection{The circuit-breaker Effect}
The contrast between governed and ungoverned behavior is illustrated in \textbf{Figure 2}. When a sudden distribution shift is injected, the ungoverned system (OFF) propagates the corruption to the decoder, resulting in high-norm output hallucinations. In contrast, the RLCS-governed system (ON) detects the shift instantaneously and triggers the \texttt{ABSTAIN} signal. This causes the gated decoder to suppress output, effectively shielding the user from invalid generations. This result confirms that **system safety is independent of component robustness**.

\subsection{Biological Domain Generalization}
The results on high-dimensional biological embeddings ($d=128$) highlight the necessity of formal calibration. Initial uncalibrated runs resulted in a 100\% rejection rate for clean data, as the Euclidean distance naturally scales with dimensionality ($\approx \sqrt{d} \approx 11.3$).

\textbf{Figure 3} demonstrates the effect of our \textit{Reference-Conditioned Calibration Layer}. By mapping raw distances to Z-scores relative to the biological reference population, we restored system utility:
\begin{itemize}
    \item \textbf{Clean Acceptance}: Increased from 0\% to 99.6\%.
    \item \textbf{Safety Retention}: Corrupted data ($\sigma=0.6$) was still rejected with 100\% accuracy.
\end{itemize}
This proves that RLCS can generalize to unfamiliar manifolds without threshold retuning, provided the calibration layer is initialized with clean reference data.

\subsection{Component Failure Envelopes}
Isolated stress tests (\textbf{Figures 4 and 5}) characterize the intrinsic volatility of individual modules. We found that the encoder lacks an internal mechanism to reject noise, leading to a linear inflation of latent variance under input stress. Furthermore, the transformer attention mechanism suffers from "entropy collapse" under localized corruption, where it fixates on noisy tokens at the expense of global context. These observations justify the resED design philosophy: since components are volatile, trust must be managed at the system level.