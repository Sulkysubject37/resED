\section{Experimental Design}

We evaluate the resED architecture through a series of stress tests and domain-transfer experiments. Our validation strategy is designed to test the limits of the system's observability and the effectiveness of its governance layer across diverse latent manifolds.

\subsection{Objective I: Observability of Representation Failure}
We first test the hypothesis that system-level failures are detectable at the representation level before they manifest as output errors.
\begin{itemize}
    \item \textbf{Setup}: We utilize a synthetic manifold with known statistics.
    \item \textbf{Perturbations}: We inject two deterministic failure modes:
    \begin{enumerate}
        \item \textit{Gradual Drift}: A slow, monotonic shift in the input mean, simulating aging sensors or environmental shifts.
        \item \textit{Sudden Shock}: A large-magnitude point perturbation, simulating localized corruption or adversarial noise.
    \end{enumerate}
    \item \textbf{Metrics}: We track the ResLik and TCS response curves to measure signal-to-noise ratio and detection latency.
\end{itemize}

\subsection{Objective II: Governance Efficacy and Suppression}
We measure the system's ability to mitigate errors via the "circuit-breaker" logic.
\begin{itemize}
    \item \textbf{Setup}: A comparative run with RLCS governance enabled (\textit{resED ON}) versus disabled (\textit{resED OFF}).
    \item \textbf{Metric}: Decoder output variance and norm during shift events.
    \item \textbf{Expectation}: The ungoverned system will produce high-variance, unpredictable outputs ("hallucinations"), while the governed system will maintain a zero-norm output during violations.
\end{itemize}

\subsection{Objective III: High-Dimensional Domain Transfer}
We evaluate the generalization of RLCS to biological data, which typically exhibits higher dimensionality and different covariance structures than vision or synthetic data.
\begin{itemize}
    \item \textbf{Dataset}: Biological gene embeddings from the Bioteque resource (128-dimensional vectors).
    \item \textbf{Challenge}: Test if the scalar thresholds calibrated on synthetic data suffer from "dimensionality collapse" (universal rejection).
    \item \textbf{Solution}: Compare uncalibrated ResLik scores against our Z-score mapped calibration layer.
\end{itemize}

\subsection{Objective IV: Component Sensitivity Characterization}
Finally, we isolate each component to define its failure envelope.
\begin{itemize}
    \item \textbf{Encoder Stability}: Measuring the L2 distortion of representations under increasing input noise.
    \item \textbf{Transformer Sensitivity}: Measuring the collapse of attention entropy when specific tokens in a sequence are corrupted.
    \item \textbf{Decoder Volatility}: Quantifying the sensitivity ratio (\(\Delta y / \Delta z\)) to establish how latent errors propagate to the final output.
\end{itemize}