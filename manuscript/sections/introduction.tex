\section{Introduction}

In the deployment of deep generative models, reliability is often treated as an attribute of the model parameters---something to be optimized via loss functions, adversarial training \cite{madry2018towards}, or calibrated via post-hoc scaling \cite{guo2017calibration}. This approach assumes that a model can be trained to be "safe" in isolation. However, empirical evidence suggests that high-dimensional neural networks are intrinsically volatile; they exhibit sensitivity to adversarial perturbations, distribution shifts, and concept drift that cannot be fully mitigated during training. When these models encounter input patterns that deviate even slightly from their training distribution, they often fail silently, producing "hallucinations" that are semantically plausible but factually groundless.

We propose a fundamental shift in perspective: \textbf{Reliability is a system property, not a component property.} 

\subsection{System Definition}
In this work, a "system" is defined not as a single end-to-end differentiable model, but as a composite of independent \textit{generators} (encoders, transformers, decoders) and \textit{regulators} (governance logic). Drawing inspiration from biological systems, which achieve robustness not through perfect components but through rigorous checkpointing and repair mechanisms---such as the p53 protein arresting cell division upon detecting DNA damage---we introduce the \textbf{resED} (Representation-Gated Encoder--Decoder) framework. 

This framework is predicated on the following core assumptions:
\begin{itemize}
    \item \textbf{Failure is Inevitable}: We assume that learned components (encoders) will inevitably produce invalid representations when exposed to out-of-distribution data. We do not attempt to train a perfect encoder; instead, we build a system that can detect and survive an imperfect one.
    \item \textbf{Components are Opaque}: We treat deep networks as opaque black boxes whose internal confidence metrics are inherently untrustworthy. A model's self-reported probability distribution is often poorly calibrated on anomalous data.
    \item \textbf{Governance is External}: Safety is enforced by a deterministic, transparent control surface that monitors the latent state. This governance is orthogonal to the learning process, allowing the safety logic to be audited independently of the model weights.
\end{itemize}

\subsection{The Limits of Model-Centric Reliability}
Prior work has largely focused on making models robust or self-aware, an approach we term "model-centric reliability." While valuable, this paradigm faces inherent limitations.
\begin{itemize}
    \item \textbf{Out-of-Distribution (OOD) Detection}: Methods like ODIN \cite{liang2018odin} and Mahalanobis distance scores \cite{lee2018mahalanobis} typically operate on model outputs or classifier logits. By treating reliability as a property of the prediction, these methods often engage too late; once the latent representation has collapsed, the system's operational integrity is already compromised.
    \item \textbf{Uncertainty Estimation}: Bayesian approximations \cite{gal2016dropout} and Deep Ensembles \cite{lakshminarayanan2017ensembles} provide confidence intervals but do not govern system behavior. A model can be "confidently wrong" if the input falls into a region where the model extrapolated incorrectly.
    \item \textbf{Robust Training}: Adversarial training \cite{madry2018towards} attempts to prevent failure by hardening the decision boundary. However, it does not provide a mechanism to manage failure when it inevitably occurs outside the training distribution.
\end{itemize}

\subsection{The RLCS Paradigm}
The \textbf{Representation-Level Control Surface (RLCS)} introduces a distinct layer of governance. It does not attempt to "fix" the model or "predict" errors. Instead, it enforces a statistical contract on the latent representation itself. By defining a "trust manifold" based on a reference population (e.g., ImageNet \cite{he2016resnet}, Bioteque \cite{fernandez2022bioteque}), RLCS converts opaque latent vectors into observable risk scores. This allows the system to distinguish between \textit{opaque components} (the engines of generation) and \textit{transparent governance} (the architects of safety).

This manuscript formalizes the resED architecture, demonstrating that a deterministic governance layer can effectively suppress hallucinations across diverse domains---from computer vision benchmarks like CIFAR-10 \cite{krizhevsky2009cifar} to high-dimensional biological embeddings---without retraining the underlying models.
