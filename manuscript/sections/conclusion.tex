\section{Conclusion}

This work establishes that reliability in deep generative models is achievable not through the pursuit of component perfection, but through the architectural enforcement of system-level governance. By defining the \textbf{resED} framework, we have demonstrated that opaque, volatile components can be safely integrated into high-stakes pipelines if they are wrapped in a transparent, deterministic control surface.

Our results confirm that the "opacity-control" paradox can be resolved: we do not need to understand \textit{why} a neural network produced a specific vector to determine \textit{whether} that vector is statistically valid. The Reference-Conditioned Calibration Layer provides the necessary translation mechanism to apply this logic across vast disciplinary gaps, from computer vision to computational biology.

Future work will focus on formalizing the theoretical bounds of the "Trust Manifold" and extending the RLCS to govern not just single representations, but complex graph-structured data. We posit that such "Complete AI Systems"—which expose their own internal state for verification—are the necessary evolution of the current paradigm.
