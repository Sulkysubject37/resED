\section{Limitations}

While the resED framework offers a robust mechanism for system-level governance, it is subject to specific operational boundaries.

\subsection{Architectural Blind Spots}
Our cross-architecture validation identified a critical limitation in Transformer-based encoders utilizing Layer Normalization. Because LayerNorm projects latent vectors onto a fixed hypersphere, pure magnitude-based perturbations (Shock) are effectively normalized away before they reach the RLCS sensors. While the system remains sensitive to directional shifts (Drift), this "Normalization Blindness" means that for certain architectures, the RLCS must be augmented with pre-normalization sensors to maintain full observability.

\subsection{Sensitivity to Optimized Perturbations}
We explicitly acknowledge that RLCS does not claim adversarial completeness. The distance-based sensors (ResLik) rely on the assumption that failure modes manifest as statistical anomalies in the latent geometry. While effective against stochastic noise and distributional drift, these sensors can theoretically be evaded by optimized adversarial perturbations designed to minimize Mahalanobis distance while maximizing semantic error. We treat adversarial robustness as a distinct, orthogonal challenge; resED provides a baseline of "natural safety" but should be paired with adversarial training for hostile environments.

\subsection{Dependency on Failure Manifestation}
The RLCS relies on the premise that semantic failure manifests as statistical anomaly in the latent space. The system guards the manifold, not the semantic meaning; a statistically "typical" representation of nonsense will theoretically result in a \texttt{PROCEED} decision, although such a vector is difficult to produce without violating the manifold constraints.

\subsection{Reference Dependency}
The governance logic is strictly conditioned on the reference population $\mathcal{P}_{ref}$. If the operational environment undergoes valid concept drift (e.g., a new biological condition emerges), the system will correctly flag it as anomalous. Distinguishing between "invalid drift" (failure) and "valid drift" (discovery) requires an external update to the reference set. The system is conservative by design; it does not "learn" to accept new data online.