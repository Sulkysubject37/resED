\section{Results}

Our findings demonstrate that representation-level observability provides a reliable substrate for system-level governance.

\subsection{Detection and Observability}
The RLCS sensors accurately capture input-level stress. As shown in Figure \ref{fig:stress_obs}, both ResLik and TCS sensors track latent drift with high monotonicity. The ResLik score provides an early-warning signal, crossing the $\tau_D=3.0$ safety threshold well before the representation is completely corrupted. This confirms that latent geometry is a high-fidelity proxy for system risk.

\begin{figure}[H]
    \centering
    \includegraphics[width=0.8\textwidth]{figures/figure1_stress_observability.png}
    \caption{RLCS Sensor Observability. ResLik and TCS scores track latent drift, triggering ABSTAIN and DEFER signals respectively.}
    \label{fig:stress_obs}
\end{figure}

\subsection{Efficacy of Gated Decoding}
Figure \ref{fig:on_off} contrasts the behavior of the governed (\textit{resED ON}) and ungoverned (\textit{resED OFF}) systems. During a sudden shock event, the ungoverned decoder produces high-variance output (hallucinations). The governed system immediately suppresses these outputs, returning a zero norm. This proves that the system's robustness is an architectural property of the governance layer, not an intrinsic feature of the model components.

\begin{figure}[H]
    \centering
    \includegraphics[width=0.8\textwidth]{figures/figure2_resed_on_off.png}
    \caption{System Response. Governance prevents hallucination by suppressing output during shock events.}
    \label{fig:on_off}
\end{figure}

\subsection{Generalization via Calibration}
The biological validation experiments highlighted the dimensionality scaling issue. Initial runs on 128-dimensional embeddings resulted in universal rejection (100\% \texttt{ABSTAIN}). Figure \ref{fig:bio_calib} shows how the Reference-Conditioned Calibration Layer restored utility. By mapping raw distances to reference-relative Z-scores, the clean acceptance rate increased to 99.6\%, while maintaining 100\% detection of high-magnitude noise.

\begin{figure}[H]
    \centering
    \begin{minipage}{0.48\textwidth}
        \centering
        \includegraphics[width=\linewidth]{figures/figure5_bioteque_calibrated_sensor_response.pdf}
        \ \ (a) Calibrated Sensor Response
    \end{minipage}
    \hfill
    \begin{minipage}{0.48\textwidth}
        \centering
        \includegraphics[width=\linewidth]{figures/figure5_bioteque_calibrated_control_distribution.pdf}
        \ \ (b) Control Distribution
    \end{minipage}
    \caption{Biological Validation. Calibration restores utility on high-dimensional data without compromising safety.}
    \label{fig:bio_calib}
\end{figure}

\begin{table}[H]
\centering
\caption{Governance Outcomes Across Domains (Acceptance Rate \%)}
\label{tab:gov_outcomes}
\begin{tabular}{lccc}
\toprule
\textbf{Condition} & \textbf{Synthetic (64D)} & \textbf{Vision (2048D)} & \textbf{Biology (128D)} \\
\midrule
Clean (Uncalibrated) & 99.8\% & 0.0\%* & 0.0\%* \\
Clean (Calibrated)   & 99.8\% & 99.7\% & 99.6\% \\
Noise ($\sigma=0.6$) & 0.0\%  & 0.0\%  & 0.0\% \\
Shock (5\%)          & 95.0\% & 95.0\% & 95.0\% \\
\bottomrule
\multicolumn{4}{l}{\footnotesize *Rejection due to dimensionality scaling mismatch.}
\end{tabular}
\end{table}

\subsection{Empirical Failure Envelopes}
Component stress testing revealed the intrinsic limits of the modules. As summarized in Table \ref{tab:failure_envelopes}, all components lack internal stability mechanisms.

\begin{figure}[H]
    \centering
    \includegraphics[width=0.6\textwidth]{figures/figure_component_resenc_stability.pdf}
    \caption{Encoder Stability. Latent distortion scales linearly with input noise.}
    \label{fig:enc_stab}
\end{figure}

\begin{figure}[H]
    \centering
    \includegraphics[width=0.6\textwidth]{figures/figure_component_restr_sensitivity.pdf}
    \caption{Transformer Sensitivity. Attention entropy collapses under heavy corruption.}
    \label{fig:tr_sens}
\end{figure}

\begin{table}[H]
\centering
\caption{Summary of Component Failure Envelopes}
\label{tab:failure_envelopes}
\begin{tabular}{lll}
\toprule
\textbf{Component} & \textbf{Observed Failure Mode} & \textbf{Impact on Output} \\
\midrule
resENC & Variance Inflation ($\Delta \text{Var} \le 1.35$) & Radial Drift \\
resTR  & Attention Collapse ($\text{Entropy} \to 2.02$) & Noise Fixation \\
resDEC & Linear Error Propagation ($S \approx 0.18$) & Direct Hallucination \\
\bottomrule
\end{tabular}
\end{table}

These results establish that while individual models are volatile, their failure modes are monotonic and observable, enabling deterministic system control.