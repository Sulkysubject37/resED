\section{Results}

Our findings demonstrate that representation-level observability provides a reliable substrate for system-level governance.

\subsection{Detection and Observability}
The RLCS sensors accurately capture input-level stress. As shown in \textbf{Figure 1}, both ResLik and TCS sensors track latent drift with high monotonicity. The ResLik score provides an early-warning signal, crossing the $\tau_D=3.0$ safety threshold well before the representation is completely corrupted. This confirms that latent geometry is a high-fidelity proxy for system risk.

\subsection{Efficacy of Gated Decoding}
\textbf{Figure 2} contrasts the behavior of the governed (\textit{resED ON}) and ungoverned (\textit{resED OFF}) systems. During a sudden shock event, the ungoverned decoder produces high-variance output (hallucinations). The governed system immediately suppresses these outputs, returning a zero norm. This proves that the system's robustness is an architectural property of the governance layer, not an intrinsic feature of the model components.

\subsection{Generalization via Calibration}
The biological validation experiments highlighted the dimensionality scaling issue. Initial runs on 128-dimensional embeddings resulted in universal rejection (100\% \texttt{ABSTAIN}). \textbf{Figure 3} shows how the Reference-Conditioned Calibration Layer restored utility. By mapping raw distances to reference-relative Z-scores, the clean acceptance rate increased to 99.6\%, while maintaining 100\% detection of high-magnitude noise.

\begin{table}[H]
\centering
\caption{Governance Outcomes Across Domains (Acceptance Rate \%)}
\label{tab:gov_outcomes}
\begin{tabular}{lccc}
\toprule
\textbf{Condition} & \textbf{Synthetic (64D)} & \textbf{Vision (2048D)} & \textbf{Biology (128D)} \\
\midrule
Clean (Uncalibrated) & 99.8\% & 0.0\%* & 0.0\%* \\
Clean (Calibrated)   & 99.8\% & 99.7\% & 99.6\% \\
Noise ($\sigma=0.6$) & 0.0\%  & 0.0\%  & 0.0\% \\
Shock (5\%)          & 95.0\% & 95.0\% & 95.0\% \\
\bottomrule
\multicolumn{4}{l}{\footnotesize *Rejection due to dimensionality scaling mismatch.}
\end{tabular}
\end{table}

\subsection{Empirical Failure Envelopes}
Component stress testing revealed the intrinsic limits of the modules. As summarized in \textbf{Table \ref{tab:failure_envelopes}}, all components lack internal stability mechanisms.

\begin{table}[H]
\centering
\caption{Summary of Component Failure Envelopes}
\label{tab:failure_envelopes}
\begin{tabular}{lll}
\toprule
\textbf{Component} & \textbf{Observed Failure Mode} & \textbf{Impact on Output} \\
\midrule
resENC & Variance Inflation ($\Delta \text{Var} \le 1.35$) & Radial Drift \\
resTR  & Attention Collapse ($\text{Entropy} \to 2.02$) & Noise Fixation \\
resDEC & Linear Error Propagation ($S \approx 0.18$) & Direct Hallucination \\
\bottomrule
\end{tabular}
\end{table}

These results, combined with the cross-architecture findings in \textbf{Figures 4 and 5}, establish that while individual models are volatile, their failure modes are monotonic and observable, enabling deterministic system control.
